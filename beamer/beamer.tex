\documentclass{beamer}
\usepackage[utf8]{inputenc}
\usetheme{Warsaw}
\usepackage{url}
\begin{document}

%%%%%%%%%%%%%%%%%%%%%%%%%%%%%%%%%%%%%%%%%%%%%%%%%%%%%%%%%%%%%%%%%%%%%%%%%%%%%%%%%%%%%%%%%%%%%%%%
\title{Accès structurés \& interopérabilités}
\author{Yoann Ricordel, \\João Prado, \\Cédric Honnet}
\institute{Télécom ParisTech\\Robotique \& Systèmes Embarqués}
\titlepage

%%%%%%%%%%%%%%%%%%%%%%%%%%%%%%%%%%%%%%%%%%%%%%%%%%%%%%%%%%%%%%%%%%%%%%%%%%%%%%%%%%%%%%%%%%%%%%%%
\section{Plan}
    \begin{frame}
        \tableofcontents
    \end{frame}

%%%%%%%%%%%%%%%%%%%%%%%%%%%%%%%%%%%%%%%%%%%%%%%%%%%%%%%%%%%%%%%%%%%%%%%%%%%%%%%%%%%%%%%%%%%%%%%%
\section{Accès structurés}

    \subsection{REST}
        \begin{frame}
            \frametitle{REST : REpresentational State Transfer}
            \begin{itemize}
            \item Elaboré en 2000 par Roy Fielding, un des créateurs du protocole HTTP
            \item Pas un potocol mais un style d’architecture pour systèmes d'information distribués :
                \begin{itemize}
                \item Une "ressource" est une chose nommable
                \item Une "représentation" est une séquence d’octets
                \item Un "composant" est un acteur, relié à d’autres composants/ressources
                \end{itemize}
            \end{itemize}
        \end{frame}

        \begin{frame}
            \frametitle{Caractéristiques}
            \begin{itemize}
            \item Stateless
            \item URI
            \item Exemple avec HTTP :
                \begin{itemize}
                \item get
                \item put
                \item post
                \item delete...
                \end{itemize}
            \end{itemize}
        \end{frame}

        \begin{frame}
            \frametitle{SOAP : Simple Object Access Protocol}
            ... :
            \begin{itemize}
                \item ...
            \end{itemize}
        \end{frame}
%%%%%%%%%%%%%%%%%%%%%%%%%%%%%%%%%%%%%%%%%%%%%%%%%%%%%%%%%%%%%%%%%%%%%%%%%%%%%%%%%%%%%%%%%%%%%%%%

    \subsection{COAP}
        \begin{frame}
            \frametitle{COAP : COnstraint Application Protocol}
        \end{frame}

%%%%%%%%%%%%%%%%%%%%%%%%%%%%%%%%%%%%%%%%%%%%%%%%%%%%%%%%%%%%%%%%%%%%%%%%%%%%%%%%%%%%%%%%%%%%%%%%
\section{Interopérabilités}

    \subsection{JSON}
        \begin{frame}
            \frametitle{JSON : JAvaScript Object Notation}
        \end{frame}

    \subsection{XML}
        \begin{frame}
            \frametitle{XML : eXtensible Markup Language}
        \end{frame}

%%%%%%%%%%%%%%%%%%%%%%%%%%%%%%%%%%%%%%%%%%%%%%%%%%%%%%%%%%%%%%%%%%%%%%%%%%%%%%%%%%%%%%%%%%%%%%%%
\section{Conclusion}

    \subsection{Résumé}
        \begin{frame}
            \frametitle{Résumé}
            \begin{itemize}
                \item Génial, j'ai fait mon 1er Beamer ! 
                \item 
            \end{itemize}
        \end{frame}

    \subsection{Références}
        \begin{frame}
            \frametitle{Références}
            \begin{itemize}
                \item \normalsize How I Explained REST to My Wife :\\
                    \tiny \url{http://www.jtomayko.com/writings/rest-to-my-wife}
                \item \normalsize L'architecture REST :\\
                    \tiny \url{http://www.journaldunet.com/developpeur/tutoriel/php/020109php_xmlrpc.shtml}
            \end{itemize}
        \end{frame}

    \subsection{?}
        \begin{frame}
            \begin{center}
            \Huge{Questions ?}
            \end{center}
        \end{frame}

%%%%%%%%%%%%%%%%%%%%%%%%%%%%%%%%%%%%%%%%%%%%%%%%%%%%%%%%%%%%%%%%%%%%%%%%%%%%%%%%%%%%%%%%%%%%%%%%
\end{document}

